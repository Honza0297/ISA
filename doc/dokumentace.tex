\documentclass[a4paper, 11pt]{article}

\usepackage[czech]{babel}
\usepackage{times}
\usepackage[text={17cm,24cm}, top=2cm, left=2cm, right=2cm, bottom=3cm]{geometry}
\usepackage[utf8]{inputenc}
\setlength{\headheight}{20.0mm}
\usepackage{fancyhdr}
\pagestyle{fancy}
\usepackage{graphics}
\usepackage {array}
\usepackage{pdflscape}
\usepackage[czech, ruled, vlined, linesnumbered, longend, noline]{algorithm2e}
\usepackage{multirow}


\begin{document}
\catcode`\-=12 %Mělo by to vyřešit problém s cline... jestli ne, tak už nevím
%Uvodni strana
\begin{titlepage}
	\begin{center}
		\Huge \textsc{Vysoké učení technické v~Brně}\\
		\huge \textsc{Fakulta informačních technologií}\\
		\vspace{\stretch{0,190}}
		\begin{figure}[h]
		\begin{center}
		\scalebox{0.85}{\includegraphics{logo.png}}
		\end{center}
		\end{figure}

		\vspace{\stretch{0,190}}
		\LARGE Projekt do předmětu ISA \\
		\Huge {Jednoduchý DNS resolver}\\
		\vspace{\stretch{0,620}}
		
		
		\Large Beran Jan \texttt{xberan43} \hspace{\stretch{0,5}} \Large\today\\
	 \end{center}
\end{titlepage}
\tableofcontents

\newpage
\section{Úvod a zadání projektu}
Zadáním projektu bylo vytvořit jednoduchý DNS resolver.  \\
Na vstupu programu se očekává adresa DNS serveru, kam zasílat dotazy (IP nebo doménové jméno) a dotazovaná adresa (opět IP nebo doménové jméno). Dále proram umožňuje specifikovat port, typ dotazu (A nebo AAAA), rekurzi na straně DNS serveru a reverzní dotazování (při zadání adresy jako IP).  \\
Na výstupu se očekávají základní informace o dotazu a dále vypsané jednotlivé části odpovědi (název, typ, třída, ttl a data).

\section{Teorie}
V této části stručně vysvětlím nastudovanou problematiku DNS dotazování a lehce nastíním způsob, jakým jsem vytvořil svůj projekt. Při studiu teorie jsem čerpal z následujícíh zdrojů:\cite{Matousek:SAR}, \cite{DNS-notes}.
\subsection{Důvod k zavedení DNS}
Jednotlivé stanice připojené k internetu jsou jednoznačně adresovány pomocí IP adres, ovšem běžní uživatelé místo nich používají textové reprezentace – doménová jména. Z toho plynoucí nutnost převodu mezi těmito dvěma způsoby adresace byla původně řešena jediným souborem na každé stanici \footnote{\texttt{/etc/hosts} na unixových systémech, stále používaný, ale nikoli hlavní způsob, jak převádět IP na doménová jména a naopak.}. Tento systém se ovšem ukázal jako nevhodný (složitá a pomalá aktualizace atd.), proto byl v roce 1983 nahrazen systémem doménových jmen - DNS. \\
\subsection{Princip DNS}
\uv{Lidsky zapamatovatelná} podoba jména je rozdělena na několik částí oddělených tečkami (například www.google.com). Jednotlivé části adresy se nazávají doménová jména n-té úrovně, přičemž doména nejvyšší úrovně (TLD, Top-Level-Domain) leží vpravo a má největší význam.\\
Informace o doménách jsou poté uloženy na tzv. jmenných serverech (DNS serverech, nameserverech, doménových serverech). Tyto informace přitom můžou být dvojího druhu: přímo IP adresa, která odpovídá danému jménu, nebo odkaz na další jmenný server, který odpovídá následující doméně. Tím je docíleno efektivní hierarchie.\\
\subsection{DNS dotazování}
DNS dotazování poté probíhá tak, že je dotaz odeslán na lokální DNS server. V tuto chvíli může nastat několik možností: server buď zná IP adresu pro dané jméno a tu vrátí tazateli, nebo zná pouze další server v hierarchii, kam dotaz směřovat. V závislosti na typu dotazu (zda je od tazatele vyžadována rekurze), je buď dotaz přímo předán dál nebo je tazateli vrácena informace o dalším serveru v pořadí.\\
\subsection{Reverzní dotazování}
Systém doménových jmen funguje i obráceně – převádí IP adresy na doménová jména. Zde je důležité si uvědomit, že části IP adresy jsou řazeny podle opačného klíče než u doménových jmen; Úplně vpraco se nachází adresa stanice v (pod)síti, vlevo adresa sítě.\\
Při DNS dotazování se tedy v dotazu neuvádí přímo původní IP adresa (např. 12.34.465.7), ale její upravená verze s koncovkou \texttt{in-addr.arpa}(7.465.34.12.in-addr.arpa). Na takto upravenou adresu se již dotazuje standardním způsobem. 

\section{Implementace}
Na základě výše uvedené teorie a zadání jsem vytvořil jednoduchý DNS resolver, který pracuje s následujícími parametry: 
\begin{itemize}
	\item -s server: Povinny argument, který specifikuje DNS server, použitá při dotazování
	\item address: adresa, která bude dotazovaná (doménové jméno nebo IP adresa) 
	\item -p port: Volitelný parametr, specifikuje port, kam odesílat dotazy. Standardní port je 53
	\item -x: Volitelný parametr, specifikuje, že se bude provádět reverzní dotazování.
	\item -r: Volitelný parametr, specifikuje, že s ebude provádět rekurzivní dotazování na straně DNS serveru (program sám vypisuje pouze první přijatou odpověď a v případě, že není tento parametr specifikován, program neprovádí iterativní dotazování ve vlastní režii).
	\item -6: Volitelný parametr, specifikuje použití dotazu typu AAAA místo standardnho A
\end{itemize}
Kromě těchto parametrů program zvládá i parametr -h, který vypíše základní informace a použití programu a ukončí program. \\

Při implemetaci jsem vycházel ze svých předešlých projektů, především kódu k druhému projektu do předmětu IPK (parsování paketů a UDP komunikace), na kterém jem také začal stavět. V průběhu implementace ovšem došlo k téměř kompletnímu přepsání. \\

Samotná implementace probíhala v iteracích, kde každá iterace měla nějaký cíl a důvod:
 \begin{enumerate}
    \item Analýza vstupních argumentů, ošetření chybných vstupů.
     \item Odesílání paketů (byť nesmyslných) \footnote{Tento na první pohled nelogický krok má svůj důvod. Odeslaný pakety jsou zaznamenávány programem Wireshark, který je umí zpětně analyzovat a tím pomáhat v budoucích krocích s případnou opravou chyb.}
     \item Vytvoření korektní DNS hlavičky
     \item Vytvoření korektního DNS dotazu
     \item Zachytávání odpovědi
     \item Její analýza a výpis
     \item Refaktorizace kódu z předchozích bodů
 \end{enumerate}

\subsection{Testování}
Program byl primárně testován ručně pomocí integrovaného vývojového prostředí Clion a jeho \uv{Debug mode}, kde jsem krokoval program a kontroloval správný tok dat a předávání řízení. V pozdějších fázích projektu jsem vytvořil i velmi jednoduchý skript v jazyce Python 3, který testoval základní testovací scénáře pro správný i nesprávný vstup.\\
Pro kontrolu komunikace s DNS serverem jsem také používal program Wireshark.\\
Dále jsem použil příkazovou utilitu dig, která mi pomáhala validovat výstup mého programu. \\

\subsection {Přejaté části kódu}
Při tvorbě projektu jsem narazil na problém při čtení některých částí odpovědi. Pro tyto účely jsem přejal funkcni ReadName() z \cite{ReadName}. Tato funkce tedy není mým dílem, jak je uvedeno i ve zdrojovém kódu. \\

\subsection{Nedokončené a nefunkční části projektu}
Projekt bohužel nezvládá zasílat dotazy typu AAAA. Zároveň také nepodporuje zadání DNS serveru pomocí IPv6 adresy.

\section{Závěr}
Nejprve proběhlo studium problematiky (zasílání DNS dotazů, programování DNS dotazů v jazyce C...), poté vlastní implementace. Výsledkem je projekt, který částečně splňuje zadání (nesplněné body zadání se dají najít v závěru předchozí sekce). 
\newpage % Použité zdroje
\bibliographystyle{czechiso}
\def\refname{Použité zdroje}
\bibliography{citaceISA}
 
\end{document}
